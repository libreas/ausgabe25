\documentclass[a4paper,
fontsize=11pt,
%headings=small,
oneside,
numbers=noperiodatend,
parskip=half-,
bibliography=totoc,
final
]{scrartcl}

\usepackage{synttree}
\usepackage{graphicx}
\setkeys{Gin}{width=.6\textwidth} %default pics size

\graphicspath{{./plots/}}
\usepackage[ngerman]{babel}
%\usepackage{amsmath}
\usepackage[utf8x]{inputenc}
\usepackage [hyphens]{url}

\usepackage[colorlinks, linkcolor=black,citecolor=black, urlcolor=blue,
breaklinks= true]{hyperref}
\usepackage{booktabs} 
\usepackage[left=2.4cm,right=2.4cm,top=2.3cm,bottom=2cm,includeheadfoot]{geometry}
\usepackage{eurosym}
\usepackage{multirow}
\usepackage[ngerman]{varioref}
\setcapindent{1em}
\renewcommand{\labelitemi}{--}
\usepackage{paralist}
\usepackage{pdfpages}
\usepackage{lscape}
\usepackage{float}
\usepackage{acronym}
\usepackage{eurosym}
\usepackage[babel]{csquotes}
\usepackage{longtable,lscape}
\usepackage{mathpazo}
\usepackage[flushmargin,ragged]{footmisc} % left align footnote

\urlstyle{same}  % don't use monospace font for urls

\usepackage[fleqn]{amsmath}

%adjust fontsize for part

\usepackage{sectsty}
\partfont{\large}

%Das BibTeX-Zeichen mit \BibTeX setzen:
\def\symbol#1{\char #1\relax}
\def\bsl{{\tt\symbol{'134}}}
\def\BibTeX{{\rm B\kern-.05em{\sc i\kern-.025em b}\kern-.08em
    T\kern-.1667em\lower.7ex\hbox{E}\kern-.125emX}}

\usepackage{fancyhdr}
\fancyhf{}
\pagestyle{fancyplain}
\fancyhead[R]{\thepage}

%meta
%meta

\fancyhead[L]{Redaktion LIBREAS \\ %author
LIBREAS. Library Ideas, 25 (2014). % journal, issue, volume.
\href{http://nbn-resolving.de/urn:nbn:de:kobv:}{urn:nbn:de:kobv:}} % urn
\fancyhead[R]{\thepage} %page number
\fancyfoot[L] {\textit{Creative Commons BY 3.0}} %licence
\fancyfoot[R] {\textit{ISSN: 1860-7950}}

\title{\LARGE{Bewerbungen von bibliothekarischen Hilfsarbeiterinnen an der Königlichen Bibliothek / Preußischen Staatsbibliothek 1916-1943}: Ein Werkstattbericht} %title %title
\author{Frauke Mahrt-Thomsen} %author

\date{}
\begin{document}

\maketitle
\thispagestyle{fancyplain} 

%abstracts

%body
Auf der Suche nach Quellen und Dokumenten zur
Bibliothekarinnen-Geschichte in Berlin stieß ich im Hausarchiv der
Staatsbibliothek zu Berlin, zugänglich über die Handschriftenabteilung
in der Potsdamer Straße, auf einen Schatz besonderer Art: Es handelt
sich um Bewerbungsunterlagen von circa 370 Frauen, die sich zwischen
1916 und 1943 um eine Anstellung in der Königlichen Bibliothek/
Preußischen Staatsbibliothek beworben haben. Die Abteilung I,
Personalakten, Untergruppe 20 des Hausarchivs enthält fein säuberlich
nach Geschlechtern getrennt die Unterlagen zu den \emph{Hilfsarbeitern}
und \emph{Hilfsarbeiterinnen} der Bibliothek.

In den zehn dickleibigen Aktenbänden zu den \emph{Hilfsarbeiterinnen}
hat die Bibliothek auch die Bewerbungsunterlagen von Frauen abgelegt,
die unter anderem als Stenotypistin, Kontoristin, Fotografin hoffen,
eine Anstellung in der Staatsbibliothek zu bekommen. Die Mehrzahl der
Bewerbungen (circa 300) stammt aber von bibliothekarisch oder akademisch
vorgebildeten Frauen, die sich um eine Stelle im mittleren
Bibliotheksdienst beworben haben.

Bis zum Ende des Zweiten Weltkrieges hielt sich insbesondere in
wissenschaftlichen Bibliotheken die Gepflogenheit, Mitarbeiterinnen, die
nach einem qualifizierten Schulabschluss (Lyze\-ums-, Obersekunda-,
Primareife, oft auch Abitur) und einer mindestens drei- bis vierjährigen
Ausbildung die Diplomprüfung \emph{für den \enquote{mittleren
Bibliotheksdienst an wissenschaftlichen Bibliotheken und den Dienst an
Volksbibliotheken und verwandten Institutionen}} bestanden haben, als
\emph{Hilfsarbeiterinnen} zu bezeichnen. Die Anforderungen für diese
Diplomprüfung wurden vom preußischen Staat erstmals durch den Erlass des
Preußischen Kulturministers vom 10.8.1909 geregelt und 1916 und 1930
modifiziert. Zu den Voraussetzungen für die Prüfung gehörten die
einjährigen Praktika sowohl in öffentlichen wie in wissenschaftlichen
Bibliotheken und in der Regel auch der Besuch einer Bibliotheksschule
und von Vorbereitungskursen in Berlin.

Obwohl die Bewerberinnen in den verschiedensten Regionen Deutschlands
und nur zu einem kleinen Teil in nichtpreußischen Ländern oder im
Ausland aufgewachsen waren, verband die meisten eine gemeinsame
Erfahrung: die schulische und kursmäßige Vorbereitung auf die
Diplomprüfung in Berlin und sehr oft auch die Absolvierung eines
Praktikums an der Königlichen Bibliothek / Preußischen Staatsbibliothek
oder an einer der anderen wissenschaftlichen und öffentlichen
Bibliotheken Berlins.

Die Frauen, deren Unterlagen in der Acta I, 20 des Hausarchivs
aufbewahrt werden, haben aufgrund ihrer damaligen Bewerbung keine
Arbeitsstelle in der Staatsbibliothek bekommen, sondern überwiegend nur
eine Nummer auf einer Vormerkliste, die immer länger wurde. Viele
erhielten aus den verschiedensten, nachfolgend aufgelisteten Gründen,
sofort einen ablehnenden Bescheid:~

\begin{itemize}
\item
  Weil sie trotz bestandener Diplomprüfung in Fächern wie Stenographie
  und~ Maschinenschreiben ein ‚mangelhaft` haben. Eine Eintragung in die
  Vormerkliste war erst nach erfolgreich absolvierter Nachprüfung für
  diese Fächer möglich.
\item
  Weil sie im Praktikum / bei der Diplomprüfung ungünstig beurteilt
  wurden.
\item
  Weil sie die sächsische Diplomprüfung in Leipzig und nicht die
  preußische in Berlin absolviert haben.
\item
  Weil sie vor dem Ersten Weltkrieg nicht in Berlin gewohnt haben
  (Antwort im August 1919) oder weil es eine Zuzugssperre nach Berlin
  gibt (Oktober 1934).
\item
  Weil sie zwar Vollakademikerinnen sind, vielleicht sogar die
  Laufbahnprüfung für~ den höheren Bibliotheksdienst gemacht haben, aber
  nicht die Diplomprüfung.
\item
  Weil sie nicht bei der Arbeitsnachweisbehörde gemeldet sind (ab 1930).
\item
  Weil sie eine Frau, jedoch Männer zu bevorzugen sind (Anweisung des
  Ministeriums ab Frühjahr 1931).
\item
  Weil es auf absehbare Zeit keine freie Stellen gibt, zumindest nicht
  in dem~ gewünschten Bereich.
\item
  Weil die Vormerklisten schon zu lang sind und für eine Weile ganz
  geschlossen werden (1936, 1939).
\end{itemize}

In einigen wenigen Fällen haben die Bewerberinnen ein Stellenangebot von
der Preußischen Staatsbibliothek erhalten, dieses aber aus verschiedenen
Gründen nicht angenommen, und anderem weil sie kurzfristig einen
attraktiveren Arbeitsplatz fanden.

Im Jahre 1939 ist die Vormerkliste bereits bei der Nummer 201 angelangt.
Nach einer Übersicht der Generaldirektion aus dem gleichen
Jahr\footnote{Personalstand der Preußischen Staatsbibliothek, in: Acta
  PrSB, Generaldirektion Hugo Andres Krüss, Nr. 263 (1939-1944).} gibt
es zu diesem Zeitpunkt 100 Frauen im mittleren Bibliotheksdienst des
Hauses. Die Vormerkliste übersteigt diese Zahl um das Doppelte. Selbst
wenn eine Reihe von Vormerkungen sich inzwischen erübrigt hatten, weil
die betreffenden Frauen inzwischen eine Stelle fanden, entweder bei der
Preußischen Staatsbibliothek oder in anderen Einrichtungen, so zeigt
diese Zahl doch den großen Bedarf und den Druck, unter dem die Frauen
bei der Suche nach einem einigermaßen akzeptablen Arbeitsplatz standen.
Viele wollten wahrscheinlich auch trotz anderer Stelle auf der
Vormerkliste der Staatsbibliothek bleiben, weil ein Arbeitsplatz dort
als attraktiver galt, mit besserer sozialer Absicherung und den
Aufstiegschancen in ein Beamtenverhältnis.~

\section*{Frauen und der Bibliotheksberuf in den
Bewerbungsakten}\label{frauen-und-der-bibliotheksberuf-in-den-bewerbungsakten}

Die Bewerbungsakten vermitteln insgesamt einen intensiven Eindruck von
der schwierigen Situation bibliothekarisch arbeitender Frauen in der
Zeit zwischen den Weltkriegen und über den fragilen Zustand der
deutschen Gesellschaft nach dem traumatischen Ende des Ersten
Weltkrieges.

Die Bewerbungsschreiben, Lebensläufe, Zeugnisse und Briefe von
Angehörigen, Freunden und Förderern sind voller Hinweise auf die Brüche,
Verwüstungen und Verunsicherungen, die der Krieg und die nachfolgenden
Krisen in den meist gutbürgerlichen Herkunftsfamilien angerichtet haben.
Sie zeigen auch, wie stark die Bewerberinnen, trotz neu gewonnener
Bildungs- und Berufsperspektiven, von ihren Familien noch in einer
auffallend geschlechterbezogenen Weise in die Pflicht genommen werden,
alten Rollenbildern zu genügen und eigene
Weiterentwicklungsmöglichkeiten zurückzustellen.

So durchziehen die Bewerbungsschreiben und Lebensläufe Hinweise auf den
\enquote{Heldentod} des Vaters, Gatten oder Verlobten, den Verlust des
Familienvermögens, die unversorgte Mutter, die noch in Ausbildung
befindlichen Geschwister, die Krankheit und Pflegebedürftigkeit der
Eltern, die Arbeitslosigkeit oder vorzeitige Pensionierung des Vaters,
die Stellungslosigkeit der Geschwister. Bewerbungen enthalten oft Sätze
wie: \enquote{Vater hat in der Inflation sein Vermögen verloren} (E. M.,
1927), \enquote{Ich muss meine Mutter und meine Schwester finanziell
unterstützen} (Ch. M., 1932) oder: \enquote{Die äußerst schwierige
wirtschaftliche Lage meiner Eltern erfordert, dass ich sofort jede
berufliche Tätigkeit ergreife.} (W. S., 1935). Sie werden zu der
wichtigsten Begründung für die Bewerberinnen, warum sie unbedingt eine
Stelle in der Staatsbibliothek haben möchten.

Nicht selten schreiben auch Väter, Freunde und hochgestellte
Unterstützer an den Bibliotheksdirektor, um ihm unter Hinweis auf die
Sorgen und Nöte der Familie die Einstellung der Kandidatin ans Herz zu
legen. So schreibt ein Oberregierungsrat B. l. zu der Bewerberin R. Sch.
1931 an die Bibliothek: \enquote{{[}\ldots{}{]} steht völlig allein und
mittellosda, hat seit Jahren eine kranke Mutter unterhalten und jetzt
eine erwerbslose Schwester.}

Manchmal wird der Druck auf die Generaldirektion zugunsten einer
bestimmten Bewerberin sehr direkt. So versucht ein Prof.~R. aus dem
Wissenschaftsministerium im Frühjahr 1930 die Generaldirektion per
Erlass anzuweisen, die Bewerberin G. Sch. einzustellen. Generaldirektor
Krüss widerspricht, willigt dann aber in einen sechswöchigen Probedienst
der Bewerberin ein.

Häufig haben die Kandidatinnen mit einigen Jahren Berufserfahrung
bereits schmerzhafte Einschnitte in ihrer Arbeitsbiographie wegstecken
müssen, den wiederholten Verlust ihres Arbeitsplatzes durch
Personalabbau und Finanzmittelkürzungen, oder durch die Verfügung, ihren
Arbeitsplatz für einen Kriegsheimkehrer freizumachen. Die Stellen, die
sie im Verlaufe der Zwanziger Jahre bekommen, sind häufig befristet, und
in der Weltwirtschaftskrise werden auch länger\-fristig Beschäftigte von
den Bibliotheken in größerer Zahl entlassen.

Ab 1930 wurde die Preußische Staatsbibliothek vom Ministerium für
Wissenschaft, Kunst und Volksbildung angewiesen, bei der Besetzung von
Stellen im mittleren Bibliotheksdienst Männer zu bevorzugen, weil Frauen
dort bereits bei weitem in der Überzahl seien. Als ein Oberregierungsrat
Schw. aus dem Wissenschaftsministerium 1930 zugunsten einer Frau v. E.
interveniert und diese sich über das ihr zu Ohren gekommene Gerücht der
Bevorzugung von Männern beschwert, kann die Bibliotheksleitung nur
bestätigen, dass dieses stimmt, aber dass es sich um eine Weisung aus
dem eigenen Ministerium handelt.

Natürlich arbeiten im mittleren Bibliotheksdienst in den Zwanziger
Jahren bereits deutlich mehr Frauen als Männer. Seit Beginn des
Jahrhunderts waren es die Frauen, die in Ermangelung anderer
Alternativen -- die akademische Ebene blieb ihnen über lange Zeit
verschlossen -- mit hoher Motivation, Bildungs- und
Leistungsbereitschaft in den mittleren Bibliotheksdienst an
wissenschaftlichen Bibliotheken und in die Volksbibliotheken strömten
und beide Bereiche maßgeblich weiter entwickelten.

Wie auch in anderen qualifizierten Tätigkeitsbereichen wurde das
Berufsfeld Bibliothek für sie nur im mittleren Bereich geöffnet, denn in
Preußen durften Frauen erst ab 1908 studieren und sich erst ab 1921 um
ein Volontariat für den höheren Bibliotheksdienst bewerben. Aber der
Zugang zur leitenden Ebene wurde nur einer sehr kleinen Anzahl von
Frauen eröffnet und so drängten sie in die mittlere Laufbahn. Das war
vielen Bibliotheksdirektoren längere Zeit durchaus sehr willkommen, weil
die weiblichen Hilfskräfte bereit waren, viele Ordnungs- und
Routineaufgaben mit größtem Fleiß bei sparsamer Entlohnung zu übernehmen
und so eine wirksame Entlastung für Bibliotheksbetrieb und das
Bibliotheksbudget darstellten.

Nach dem Ersten Weltkrieg baut sich unter den leitenden
Bibliotheksdirektoren eine schrittweise Fronde gegen die Feminisierung
der mittleren Bibliotheksebene auf. So sendet Fritz Milkau, seit 1921
Nachfolger Adolf von Harnacks als Generaldirektor der Preußischen
Staatsbibliothek, seinem Minister am 18.9.1923 eine Stellungnahme
\enquote{betr. die Verwendung von Frauen im mittleren
Bibliotheksdienst.}\footnote{Schreiben des Vorsitzenden des Beirats für
  Bibliotheksangelegenheiten, Fritz Milkau, an den Minister für
  Wissenschaft, Kunst und Volksbildung, vom 18.9.1923, Geheimes
  Staatsarchiv Preußischer Kulturbesitz, UIK 8853, Bl.42.} Er vermeint
deutliche Grenzen ihrer physischen und psychischen Belastbarkeit zu
sehen, weil sie nicht \enquote{lange Stunden hintereinander
{[}\ldots{}{]} stehend mit schweren Bänden hantieren} könnten oder
\enquote{weil sie leichter die Ruhe verlieren}, eine größere
Anfälligkeit für Krankheiten hätten und natürlich würde es zu
Unzuträglichkeiten kommen, wenn man auch nur den Versuch machen würde,
\enquote{weibliche Beamte zu Vorgesetzten von männlichen zu machen.} Er
plädiert deshalb bereits 1923 für eine stärkere Maskulinisierung des
mittleren Dienstes.

Als in der großen Wirtschaftskrise immer mehr Männer arbeitslos werden,
zögert man auf höherer ministerieller Ebene nicht länger, daraus 1930
ein verbindliches Dekret zu machen: Männer sind, nicht nur bei gleichen
Qualifikationen, sondern grundsätzlich und überhaupt, bei
Stellenbesetzungen im mittleren Bibliotheksdienst vorzuziehen.

Wenn die geschlechterdiskriminierende Anweisung hier noch in der Form
einer offiziellen Anweisung vorgenommen wurde, so geschah es auf der
Ebene des höheren Bibliotheksdienstes in einer wesentlich versteckteren
Art und Weise. Der Zugang von Frauen zum Höheren Bibliotheksdienst wurde
von Anfang an sehr restriktiv gehandhabt und phasenweise de facto
unterbunden. Nach den Untersuchungen von Dagmar Jank\footnote{Dagmar
  Jank: Frauen im höheren Bibliotheksdienst, in: Verein Deutscher
  Bibliothekare 1900-2000. Festschrift. Wiesbaden 2000, S.302ff.}
erhielten zwischen 1921 und 1938 45 Frauen die Zulassung als
Volontärinnen für den höheren Bibliotheksdienst, davon beendeten 37 ihre
Ausbildung und arbeiteten überwiegend bis zu ihrer Pensionierung in
wissenschaftlichen Bibliotheken -- wenn sie nicht aufgrund des
\emph{Gesetzes zur Wiederherstellung des Berufsbeamtentums} vom 7.4.1933
wegen ihrer jüdischen Herkunft entlassen wurden wie Clara Stier-Somlo,
Helene Wieruszowski und Anneliese Modrze, die alle drei zeitweise oder
bis zum Schluss bei der Preußischen Staatsbibliothek gearbeitet haben.

Einige wenige Frauen schafften also den Zugang zu der höheren
Bibliothekslaufbahn, aber als in der 35. Sitzung des Preußischen Beirats
für Bibliotheksangelegenheiten erneut über die Zulassung von Volontären
und Volontärinnen gesprochen wurde, plädierte Hugo Andres Krüss, seit
1925 Generaldirektor der Preußischen Staatsbibliothek, dafür,
\enquote{größere Zurückhaltung bei der Annahme von weiblichen Bewerbern
zu üben}. Der Vertreter des Kultusministeriums gab dann vor, um die
\enquote{Führerinnen der Frauenbewegung}, die \enquote{schon jetzt
nervös} seien, nicht weiter zu provozieren, solle man
\enquote{stillschweigend} für ein bis zwei Jahre keine Frauen zulassen.
Die fünfzehn männlichen Mitglieder des Beirats erhoben keinen
Widerspruch.\footnote{Zitiert nach Erwin Marks: Aus dem Protokoll
  geplaudert. In: Laurentius (1992), s.123.}

In den Bewerbungsakten findet sich das Beispiel von Dr.~Edith Adelheid
Rothe (Jg. 1897), die 1925 promoviert und 1927 die Prüfung für den
Höheren Dienst an wissenschaftlichen Bibliotheken ablegte. 1934 unterzog
sie sich zusätzlich der Diplomprüfung für den mittleren
Bibliotheksdienst, weil ihr offenbar deutlich gemacht wurde, dass sie
nur so eine Chance hätte, überhaupt eine Stelle im Bibliothekswesen zu
bekommen. Sie wird von Krüss in die Vormerkliste für die
Hilfsarbeiterinnen eingetragen. Selbst diese Eintragung wird anderen
Akademikerinnen, die auch die Diplomprüfung absolviert haben, aber bei
der immer noch vorgeschriebenen Prüfung in Stenographie und
Maschinenschreiben nur ein mangelhaft vorweisen konnten, rigoros
verweigert (Bsp. Dr.~Jenny Müller, Jg 1895) oder erst vorgenommen, wenn
die Bescheinigung über die gelungene Nachprüfung eingereicht wird
(Dr.~Hildegard Lullies, Jg. 1902).

Einer anderen Bewerberin, die kurz vor der Promotion steht und sich nach
der Möglichkeit einer späteren Anstellung in der Staatsbibliothek
erkundigt, wird 1937 mitgeteilt, dass man \enquote{jetzt oder später
keine Möglichkeit sieht, sie an der Staatsbibliothek zu beschäftigen},
da nur Kräfte mit bibliothekarischer Ausbildung genommen werden, und man
sagt ihr in aller Deutlichkeit: \enquote{Die wissenschaftliche Laufbahn
bietet für Frauen zur Zeit keine Aussicht}! Die wenigen, bereits im
höheren Dienst befindlichen Frauen konnten bleiben (in der Preußischen
Staatsbibliothek z.B. Dr.~Käthe Iwand und Dr.~Luise von Schwartzkoppen),
aber ein Neuzugang von Frauen zum höheren Bibliotheksdienst war
offensichtlich nicht mehr möglich.

\section*{Lebensläufe der
Bewerberinnen}\label{lebensluxe4ufe-der-bewerberinnen}

Zurück zu der Liste der Bewerberinnen für eine Stelle im mittleren
Bibliotheksdienst. Bei einem Blick auf den meistens angegebenen Beruf
des Vaters fällt auf, wie durchgängig die Bewerberinnen für eine mäßig
bezahlte Hilfsarbeiterinnen-Stelle aus einem ausgesprochen gut- bis
großbürgerlichen Milieu stammten. Die Väter sind Landgerichts- und
Regierungsräte, Rechtsanwälte, Ärzte, Lehrer, Pfarrer, Professoren,
Studienräte und Bibliotheksdirektoren, Fabrikbesitzer, Leutnante und
Landwirte, Kartographen, Redakteure und Kaufleute, manchmal auch
mittlere Beamte, Zollsekretäre, Postinspektoren, Buchbinder und
Stadtgärtner. Proletarische Väter kommen nicht vor, auch kaum kleine
Angestellte und Gewerbetreibende, keine Musiker, Maler oder andere
Kreative, auch keine Techniker oder Ingenieure. Es ist eine sehr
homogene, fast zu sehr in sich abgeschlossene Herkunftsschicht, die den
beruflichen Nachwuchs aus den eigenen, am preußischen Beamtentum
orientierten Kreisen rekrutiert.

Eine interessante Besonderheit sind die bibliothekarischen
Herkunftsfamilien, die Fälle, in denen Töchter, Schwestern oder sonstige
Verwandte von Bibliothekaren den bibliothekarischen Beruf ergriffen oder
die eheliche Verbindung mit einem Bibliothekar gesucht haben. Schon in
der ersten Generation der Bibliothekarinnen gibt es dafür einige
markante Beispiele wie Anna Reicke (Tochter des Oberbibliothekars
Prof.Dr.~Reicke aus Königsberg), Anna Harnack und Martha Schwenke
(Töchter des Generaldirektors Adolf von Harnack und des Ersten Direktors
Dr.~Paul Schwenke an der Königlichen Bibliothek).

In den \emph{Hilfsarbeiterinnen}-Akten stößt man noch auf eine ganze
Reihe weiterer Bewerberinnen aus Bibliothekarsfamilien:

\begin{itemize}
\item
  \emph{Ursula Altmann}, Tochter des Leiters der Musikabteilung der
  Königlichen Bibliothek, Prof. Dr. Wilhelm Altmann
\item
  \emph{Margarethe Fritz}, Schwester von Dr. Gottlieb Fritz, dem Leiter
  der Stadtbibliothek Charlottenburg und Direktor der Berliner
  Stadtbibliothek
\item
  \emph{Roswitha Fritz}, verh. Kohler, Tochter von Gottlieb Fritz
\item
  \emph{Charlotte Goldschmidt}, geb. von Orth, Ehefrau des
  Bibliotheksrats an der UB Münster, Dr. Günther Goldschmidt
\item
  \emph{Margarethe Günther}, Tochter des Direktors der Danziger
  Stadtbibliothek, Prof. Dr. Otto Günther
\item
  \emph{Hildegard Karsten}, Nichte des Direktors der Lippischen
  Landesbibliothek, Dr. Ernst Anemüller
\item
  \emph{Luise Kopfermann}, Tochter des Oberbibliothekars Dr.(?)
  Kopfermann an der Königlichen Bibliothek
\item
  \emph{Marie-Luise Notzke}, Tochter des Oberbibliothekars Johannes
  Notzke, Leiter der Reichsbank-Bibliothek
\item
  \emph{Renate Stier}, Tochter des Reichstags-Bibliothekars Dr.jur.
  Gerhard Stier
\item
  \emph{Maria Luise Trommsdorff}, Tochter des Oberbibliothekars an der
  TH Hannover Dr. Paul Tromms\-dorff
\item
  \emph{Gertrud Wille}, Tochter des stellvertretenden Direktors der UB
  Berlin Dr. Wille
\end{itemize}

Nach dem Besuch der Höheren Töchter- oder Mädchenschule, später des
Lyzeums und Oberlyzeums und dem Besuch einer Handelsschule oder
Frauenschule folgt in einer nicht geringen Anzahl von Lebensläufen der
Hinweis auf den Besuch eines Hauswirtschaftskursus oder auf eine
Hauswirtschaftslehre. Auch als Erwachsene verbringen viele Bewerberinnen
längere Phasen im Elternhaus, um sich nach dem Tode eines Elternteils um
den verbliebenen Elternteil zu kümmern, ihm Gesellschaft zu leisten und
für den Haushalt zu sorgen. So führt Elise F. (Jg.1891) nach dem Tod der
Mutter für volle zwölf Jahre (1909-21) den Haushalt des Vaters und lebt
Ilse P. (Jg.1905) von 1922-24 im Haushalt der Eltern und beginnt erst
dann mit der Berufsausbildung.

Bei den Angaben zu ihrer Schulzeit ist auffallend, wie häufig die
Bewerberinnen -- in Abhängig\-keit vom Beruf des Vaters und den
Ereignissen der Zeitgeschichte -- den Wohn- und Schulort gewechselt
haben. Manche wachsen zunächst in den östlichen Gebieten Preußens auf,
im Warthegau / Westpreußen, Oberschlesien und Memelland. Sie müssen mit
ihren Familien nach dem Ende des Ersten Weltkrieges aus diesen Gebieten
fliehen, werden zur Minderheit unter der polnischen Regierung oder unter
internationale Verwaltung gestellt. Andere fliehen vor den
revolutionären Veränderungen in Russland oder Spanien und erreichen
Deutschland zum Teil erst auf langen Umwegen. Eine Umsiedlerin kommt
durch den Hitler-Stalin-Pakt (1939) aus dem Baltikum ins Deutsche Reich.
Für alle diese historischen Umbrüche finden sich Beispiele in den
\emph{Hilfsarbeiterinnen}-Akten. So schreibt E. G. (Jg.1910) in ihrer
Bewerbung, dass sie in Konitz/Westpreußen aufwuchs, \enquote{bis mein
Vater starb und wir vor den Polen flüchten mussten.} Die Familie G.
wiederum bleibt im Memelland, obwohl es 1919 nach den Bestimmungen des
Versailler Vertrages unter alliierte Verwaltung gerät und 1923 von
Litauen annektiert wird. Der Vater engagiert sich politisch und wird zum
Führer der Memelländischen Landwirtschaftspartei, der stärksten Partei
der deutschsprachigen Bevölkerung. Seine Tochter absolviert die
bibliothekarische Ausbildung in Berlin und als sie kurz vor der Prüfung
steht, sucht im März 1932 ein Attaché des Auswärtigen Amtes persönlich
den Generaldirektor der Königlichen Bibliothek auf, um ihm den Wunsch
des Führers der Memelländischen Landwirtschaftspartei nach einer Stelle
für seine Tochter zu überbringen. Krüss antwortet, dass er sehr wohl
verstehe, \enquote{welches Interesse daran besteht, Herrn G. eine
Freundlichkeit zu erweisen.} Er wäre gern bereit, daran mitzuwirken,
doch er hätte keine Möglichkeit dazu, \enquote{sofern nicht ein Ihnen
zur Verfügung stehender Fonds dazu herangezogen werden könnte.}

Ich vermute, dass das Auswärtige Amt nicht bereit und in der Lage war,
der Preußischen Staatsbibliothek eine Stelle zu schenken.

Zu den bibliothekarischen Ausbildungsverläufen der Bewerberinnen, zu der
Art und Weise, wie sie die in Berlin jeweils verfügbaren
Bibliotheksschulen und -kurse mit Praktika in allen Teilen Preußens, mit
den verschiedensten Privatstudien, Arbeits- und Auslandserfahrungen
kombiniert haben, um sich auf die Diplomprüfung \emph{für den mittleren
Bibliotheksdienst an wissenschaftlichen Bibliotheken und den Dienst an
Volksbibliotheken und verwandten Institutionen} vorzubereiten und
andererseits ihren vielfältigen Interessen zu folgen, wäre ein weiterer
Werkstattbericht nötig.

Ein Seitenblick wäre dabei angebracht auf die nicht ganz kleine Zahl
bibliothekarischer Bewerberinnen, die zwischendurch glaubten, ihr
dauerhaftes Glück in der Ehe zu finden, ihren Arbeitsplatz aufgaben,
Kinder bekamen und dann plötzlich von unvorhergesehenen Ereignissen
betroffen werden: Der Mann stirbt oder er verliert seine Arbeit oder er
lässt sich scheiden und zahlt keinen Unterhalt --~ unter Umständen, weil
er selber keine Gehalts- oder Pensionszahlungen mehr bekommt. Die Frau
hat eventuell vor Jahren ihre Diplomprüfung bestanden, aber kaum
Berufserfahrung und muss sich unter äußerst ungünstigen Umständen,
vielleicht mitten in der Wirtschaftskrise, wieder einen Arbeitsplatz
suchen. Solche dramatischen Lebensverläufe finden sich in nicht geringer
Zahl in den Bewerbungsunterlagen oder Bittbriefen von Freunden und
Förderern, die den Generaldirektor der Preußischen Staatsbibliothek
erreichen.

\section*{Nationalsozialismus}\label{nationalsozialismus}

Die NS-Zeit findet zunächst mit Verzögerung, dann aber deutlich ihren
Niederschlag in den Bewerbungsakten. Ab 1934/35 häufen sich die
NS-konformen Grußformeln und Loyalitätsbeweise in den
Bewerbungsunterlagen und die Bibliothek zögert nicht, den Anforderungen
der neuen Zeit Genüge zu tun. Ab August 1933 wird von den Bewerberinnen
die Unterzeichnung einer vorformulierten Arier-Erklärung verlangt, ab
Dezember 1933 wird dafür ein Formblatt ausgegeben. 1934 unterzeichnen
sowohl eine Bewerberin wie ein Bibliotheksdirektor \enquote{mit
deutschem Gruß}. Die Tochter eines Pfarrers rühmt sich, dass sie
Mitglied der NSBO (Nationalsozialistischen Betriebsorganisation) der TH
Berlin und der DAF (Deutsche Arbeitsfront) ist, ab 1935 verwenden sowohl
die Bewerberinnen wie die Bibliotheksdirektoren regelmäßig die
Grußformel \enquote{Heil Hitler!}. Die Bibliothekarin I. H. berichtet,
dass ihre Mutter schon seit 1931 in der NSDAP ist und der ältere Bruder
leider ohne Stellung, aber Rottenführer in der SA. Im gleichen Jahr sind
es zwei Bewerberinnen mit Promotion, die in besonderer Weise ihre
Loyalität bekunden. Dr.~H. B. erklärt freudig, \enquote{daß ich alte
Parteigenossin bin}, und Dr.~R. W. beginnt ihre Bewerbung mit den
Worten: \enquote{Ich bin 27 Jahre alt und rein arischer Abstammung.}

Diese Art Einstieg macht Schule in den Bewerbungsschreiben. Ab 1937
betonen immer mehr Bewerberinnen, dass sie Mitglied der Partei, des BDM,
der NS-Volkswohlfahrt und -Frauenschaft sind und beteuern zunehmend auch
ihre feste Verankerung in der evangelischen Kirche: \enquote{Ich bin
evangelischer Religion und rein arischer Abstammung},\enquote{Ich bin 25
Jahre alt, evangelischer Konfession und arischer Abstammung}
oder\enquote{Ich bin arischer Abstammung, deutscher Staatsangehörigkeit
und evangelisch-lutherischen Bekenntnisses.}

1938 versichert die Mädelschaftsführerin I. L. R., \enquote{dass meine
Ahnen bis vor 1800 rein arischer Abstammung waren} und im gleichen Jahr
beteuert H. K. \enquote{Ich bin 42 Jahre alt und politisch durchaus
zuverlässig.} 1941 schreibt die Bewerberin H. G. S. aus Wien:
\enquote{Ich werde meinen Vorgesetzten unbedingten Gehorsam leisten.
Heil Hitler!}

Zwar scheint sich die Stellensituation für bibliothekarisch arbeitende
Frauen im Verlauf der Dreißiger Jahre etwas entspannt zu haben, da die
bestehenden Einrichtungen schrittweise wieder mit besserer Finanzierung
rechnen konnten und zentral und dezentral neue Einrichtungen geschaffen
werden, die auch Bibliothekarinnen Arbeitsplätze bieten. So nennen die
Bewerberinnen folgende Institutionen, in denen sie nach 1933 eine
Arbeitsstelle fanden:

\begin{itemize}
\item
  Reichsstelle für das Volksbüchereiwesen (Ltr. Fritz Heiligenstaedt)
\item
  Reichsjugendbücherei
\item
  Wehrkreisbüchereien, mindestens 18 (Nr.18 befand sich in Salzburg)
\item
  Kirchenbuchamt, 1936 neu eingerichtet für die Sippenforschung
\item
  Pressearchiv des Reichsproganda-Ministeriums
\end{itemize}

Nach Kriegsbeginn kommen neue Aufgaben hinzu: die Verwaltung der
Bibliotheken in den besetzten Ländern oder ihre Ausplünderung und die
Sichtung und Aufbereitung des Raubguts in heimischen Depots und
ähnlichem. Der Überhang auf der Vormerkliste der Staatsbibliothek
schmilzt offenbar rasch, nach 1939 gibt es nur noch einige wenige
Bewerbungsunterlagen in den Akten.

Die letzte, zu den Akten gelegte Bewerbung stammt vom 17.3.1943
(Anmerkung der Verfasserin: Das war der Tag meiner Geburt!). Die Antwort
der Preußischen Staatsbibliothek lautet, dass die
\enquote{Neueinstellung von Personal nicht mehr zulässig ist.}

\begin{center}\rule{3in}{0.4pt}\end{center}

\textbf{Frauke Mahrt-Thomsen}, Jahrgang 1943, aufgewachsen in
Schleswig-Holstein, 1964-67 Ausbildung zur Diplom-Bibliothekarin in
Berlin, 1967-2008 Tätigkeit an der Stadtbibliothek
(Friedrichs\-hain-)Kreuzberg, 1975-2002 als Leiterin einer
Stadtteilbibliothek, 1988-2011 Mitglied von Akribie (Arbeitskreis
Kritischer BibliothekarInnen), ab 2011 Arbeitskreis Kritische
Bibliothek, seit 2008 Mitarbeit im Netzwerk der deutschsprachigen
Frauenarchive und -bibliotheken.

%autor

\end{document}
