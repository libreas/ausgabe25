\documentclass[a4paper,
fontsize=11pt,
%headings=small,
oneside,
numbers=noperiodatend,
parskip=half-,
bibliography=totoc,
final
]{scrartcl}

\usepackage{synttree}
\usepackage{graphicx}
\setkeys{Gin}{width=.6\textwidth} %default pics size

\graphicspath{{./plots/}}
\usepackage[ngerman]{babel}
%\usepackage{amsmath}
\usepackage[utf8x]{inputenc}
\usepackage [hyphens]{url}
\usepackage{booktabs} 
\usepackage[left=2.4cm,right=2.4cm,top=2.3cm,bottom=2cm,includeheadfoot]{geometry}
\usepackage{eurosym}
\usepackage{multirow}
\usepackage[ngerman]{varioref}
\setcapindent{1em}
\renewcommand{\labelitemi}{--}
\usepackage{paralist}
\usepackage{pdfpages}
\usepackage{lscape}
\usepackage{float}
\usepackage{acronym}
\usepackage{eurosym}
\usepackage[babel]{csquotes}
\usepackage{longtable,lscape}
\usepackage{mathpazo}
\usepackage[flushmargin,ragged]{footmisc} % left align footnote

\usepackage{listings}

\urlstyle{same}  % don't use monospace font for urls

\usepackage[fleqn]{amsmath}

%adjust fontsize for part

\usepackage{sectsty}
\partfont{\large}

%Das BibTeX-Zeichen mit \BibTeX setzen:
\def\symbol#1{\char #1\relax}
\def\bsl{{\tt\symbol{'134}}}
\def\BibTeX{{\rm B\kern-.05em{\sc i\kern-.025em b}\kern-.08em
    T\kern-.1667em\lower.7ex\hbox{E}\kern-.125emX}}

\usepackage{fancyhdr}
\fancyhf{}
\pagestyle{fancyplain}
\fancyhead[R]{\thepage}

%meta
%meta

\fancyhead[L]{M. Metzendorf, A. Kellersohn \\ %author
LIBREAS. Library Ideas, 25 (2014). % journal, issue, volume.
\href{http://nbn-resolving.de/urn:nbn:de:kobv:11-100219308
}{urn:nbn:de:kobv:11-100219308}} % urn
\fancyhead[R]{\thepage} %page number
\fancyfoot[L] {\textit{Creative Commons BY 3.0}} %licence
\fancyfoot[R] {\textit{ISSN: 1860-7950}}

\title{\LARGE{Frauen in bibliothekarischen Führungspositionen – Ein Gespräch im Mai 2014}} %title %title
\author{Maria-Inti Metzendorf \& Antje Kellersohn} %author

\setcounter{page}{67}

\usepackage[colorlinks, linkcolor=black,citecolor=black, urlcolor=blue,
breaklinks= true]{hyperref}

\date{}
\begin{document}

\maketitle
\thispagestyle{fancyplain} 

%abstracts

%body
\textbf{Frau Kellersohn, wir haben uns vor etwa 12 Jahren kennengelernt,
als ich mitten im Studium steckte und Sie an der Fachhochschule
Darmstadt am Studiengang Informations- und Wissensmanagement als
Gastdozentin das Seminar \enquote{Bibliotheksmanagement} hielten. Dieses
ist mir sehr positiv in Erinnerung geblieben, da es von Ihnen
abwechslungsreich und praxisnah gestaltet wurde. Sie stellten sich
damals als promovierte Chemikerin vor, die Leiterin der Bielefelder
Fachhochschulbibliothek war. Seit dem 1. Oktober 2008 sind sie nun
Leiterin der Universitätsbibliothek Freiburg.}

Ich habe relativ kurz entschlossen mit meiner Familie den Wechsel nach
Freiburg angetreten. Das war nicht von langer Hand vorbereitet. Ich war
gar nicht auf einen Wechsel aus. Ich bin seitens der Universität
Freiburg angesprochen worden, ich solle mich doch bewerben. Das habe ich
gemacht. Dann kam die Zusage und ich bin in einer relativ kurzen Zeit
gewechselt.

\textbf{Mit Familie.}

Ja, das war mir auch wichtig. Ich habe von Anfang an klar gemacht, dass
es mich nicht alleine gibt. Ich hatte schon jahrelang eine
Wochenendbeziehung gepflegt und das war genug. Die Uni Freiburg hat sich
sehr reingehängt. Hier gibt es einen sogenannten Dual Career Service,
der konnte ein wenig vermittelnd tätig werden. Nicht, dass sie meinem
Mann -- er ist im Schuldienst tätig -- eine Stelle besorgt hätten. Aber
sie konnten im Schulamt ein bisschen auf die Dringlichkeit der
Versetzung pochen und so ging das alles etwas schneller. Das war für uns
sehr schön, auch für mich ganz persönlich. Wenn ich merke: Da ist ein
neuer Arbeitgeber, dem dieses Thema auch am Herzen liegt. Der sich darum
kümmert und mir auch ein paar Adressen schickt, wo man Kinderbetreuung
organisieren oder eine Wohnung finden kann. Das hilft einem schon. Wir
vermitteln dort auch laufend neue Mitarbeiter hin, Kolleginnen und
Kollegen, die ebenfalls Hilfestellung bekommen. Es hat bei vielen noch
etwas anrüchiges, gerade im Wissenschaftsbereich hier in Deutschland. In
den USA ist das seit Jahrzehnten gang und gäbe. Wenn man dort einen
Professor beruft, egal ob männlich oder weiblich, weiß man, da kommt in
der Regel noch Anhang mit, für den man auch etwas organisieren muss.

\textbf{Man sieht den Menschen also nicht nur in seiner Funktion als
Arbeitskraft, sondern auch in seinem Umfeld.}

Genau. Und weiß, wenn man ihm da den Weg ebnet, dann arbeitet er
produktiver, motivierter und er bleibt auch lieber da.

\textbf{Wie kamen Sie eigentlich dazu, als Chemikerin ins
Bibliothekswesen zu wechseln? Was hat Sie an dem Ort Bibliothek
gereizt?}

Das ist ja eigentlich die klassische Frage. Ich frage mich immer, ob sie
mir auch gestellt würde, wenn ich Historikerin wäre. In meinem Beisein
ist diese Frage jedenfalls noch nie einem Geisteswissenschaftler
gestellt worden \ldots{}

Die Antwort ist eigentlich ganz einfach: Ich habe schon immer viel in
Bibliotheken gearbeitet. Auch schon als Kind -- vielleicht geprägt durch
meine Eltern als regelmäßige Bibliotheksnutzer -- habe ich fast in der
Stadtbibliothek lesen gelernt, war immer Nutzerin und habe dort viel
Zeit verbracht. In der Schulzeit, ab der 8. oder 9. Klasse, war ich dann
auch regelmäßiger Nutzer der Universitätsbibliothek vor Ort, die mich
auch ein stückweit ins Studium gebracht hat. Was man heute so unter dem
Stichwort Übergang Schule-Studium und Informationskompetenz propagiert,
habe ich damals ohne besonderes Zutun des Personals dort praktiziert,
habe aber umgekehrt auch immer die Bibliothek als sehr offenen und mir
zugewandten Ort erlebt, in dem ich auch willkommen war. Da hat niemand
geguckt und gefragt: Studierst du denn auch hier? Bist du hier Nutzer?
Darfst du dieses oder jenes nutzen? Das war für mich ein Ort des
Eintauchens, des Lesens, des Entdeckens. Keiner hat mir vorgeschrieben
welche Bücher ich aus dem Regal ziehen darf. Das war toll!

\textbf{Sozusagen eine Oase.}

Ja, eine Oase. Bibliotheksarbeit wurde auch im Studium fest verankert,
vermittelt und erwartet. Dass man abends nach den Veranstaltungen, dem
Praktikum und später nach der Laborarbeit in die Bibliothek ging, war
selbstverständlich. Um Zeitschriftenartikel zu lesen, sich auf dem
Laufenden zu halten, in den Chemical Abstracts seine Recherchen zu
machen, den Beilstein und den Gmelin durchzuackern, dazu brauchte man
die Bibliothek. Das ließ sich noch nicht online machen und so habe ich
jede Woche einige Abende in der Bibliothek verbracht. Anders als manche
Geisteswissenschaftler das von Naturwissenschaftlern erwarten. Und dann
kam noch eins drauf. Als ich mit meiner Dissertation begann, habe ich
mich in ein neues Arbeitsfeld einarbeiten müssen. Ich habe in meinem
Promotionsprojekt nanokristallines Kupfer hergestellt und mit diesem
viele Untersuchungen durchgeführt. Über die dafür nötigen Apparaturen
hatte man in diesen Jahren -- das war in den frühen 90ern -- noch kaum
verbrieftes Wissen in gängiger Fachliteratur. Ich war angewiesen auf
japanische Patentschriften! Wenn Sie jetzt mal denken, das alles vor 20
Jahren\ldots{} Ich war steter Gast in der Bibliothek und wenn ich kam,
ging so mancher Bibliothekar doch ganz dringend in die Mittagspause.
\emph{(lacht)}

\textbf{Ich verstehe.}

So war die Bibliothek für mich wieder ein Ort, wo mir geholfen wurde. Wo
mir Wissen und Information zur Verfügung gestellt wurde, das für mich in
keiner anderen Weise verfügbar war. Das mir geholfen und viele
Fehlschläge erspart hat. Da bin ich dann neugierig geworden und habe mir
überlegt: Wer arbeitet da eigentlich? Welche Qualifikationen haben die
Leute? Und habe dann natürlich auch mit dem zuständigen Fachreferenten
gesprochen. Ich erfuhr, dass man ein Referendariat machen kann. Das
Auswahlverfahren ist ein hartes, sagte man mir. Man steht in Konkurrenz
mit allen Fachdisziplinen. Ich solle es aber einfach ausprobieren, auch
wenn ich meine Arbeit noch nicht abgeschlossen habe. Im ersten Anlauf
klappe das nie. Es hat dann doch geklappt und ich bekam 1992 meinen
Referendariatsplatz in Heidelberg zugewiesen. Meine experimentellen
Arbeiten im Labor waren glücklicherweise abgeschlossen, so bin ich an
die Bergstraße gezogen und habe etwas später auch meine Arbeit
abgegeben. Ich habe den Schritt nie bereut.

Ich glaube, damit ist ein bisschen erklärt, warum ich im
Bibliothekswesen gelandet bin. Was mich auch gereizt hat an der Arbeit,
war sicherlich noch mein Gefühl. Es war wahrscheinlich wirklich nur ein
Bauchgefühl. Da tut sich was im IT-Bereich! Es gab ja noch kein Internet
wie wir das heute kennen. Aber dadurch, dass ich in einem Umfeld
gearbeitet habe, in dem man schon früh auf die IT angewiesen war -- in
der physikalischen Chemie, wir haben auf Großrechnern gearbeitet, auf
der Crey \emph{(einem Supercomputer)} gerechnet, wir haben
Datentransporte an Großrechenzentren gemacht, ich habe schon E-Mails
verschickt, FTP-Dienste genutzt -- da habe ich gemerkt, da ist etwas
kurz vorm explodieren\ldots{} Und ich habe gedacht: Das muss es sein!
Wenn man im Informationsbereich tätig werden kann, an vorderster Front
mitschwimmen und gleichzeitig noch einen Transfer zwischen den
Wissenschaftsdisziplinen miterleben und gestalten kann\ldots{} Genau das
hat sich bewahrheitet. Wir haben diese Revolution durch die
elektronischen Medien und haben heutzutage durch das Internet völlig
neue Arbeitsweisen. Das macht einfach Spaß! Wenn das nicht so gekommen
wäre und wir weiterhin Bibliotheken so betreiben würden, wie es in der
ersten Hälfte des letzten Jahrhunderts war, dann hätte ich
wahrscheinlich schon längst einen anderen Job.

\textbf{Ja, das ist gut nachvollziehbar. Und wenn Sie ein promovierter
Chemiker gewesen wären, hätten Sie sich dann auch für das
Bibliothekswesen entschieden?}

Das muss ich mit einem uneingeschränkten Ja beantworten. Das war keine
frauenspezifische Entscheidung, mitnichten.

\textbf{Sie haben sich also nicht dafür entschieden, weil sie den
Öffentlichen Dienst als reizvollen Arbeitgeber empfunden haben.}

Nein.

\textbf{In einem Artikel über Sie in der Badischen Zeitung, der kurz
nach Ihrem Wechsel nach Freiburg erschien, habe ich gelesen, Sie wollten
auch nicht unbedingt in der chemischen Industrie landen.}

Das ist richtig, hat aber auch nichts mit dem Geschlecht zu tun. Das war
eine sehr frühe Entscheidung. Schon im Grundstudium haben wir eine
Exkursion zur BASF nach Ludwigshafen gemacht. Einen Tag lang waren wir
da. Das Schlüsselerlebnis war für mich der Besuch der Essigsäurenstraße.
Die Straße heißt wirklich so. Da wird die Essigsäure produziert, eine
Riesenanlage. Der Betriebsleiter dieser Produktion -- das war so ein
richtig gestandener Mann mit vielen Jahrzehnten Berufserfahrung -- der
stand kurz vor der Rente und freute sich auch darauf. Er erzählte uns
von seinem Lebenswerk: Er hatte die Produktionsrate von 94,3\% auf
94,4\% erhöht. Da war der Mann stolz drauf. Zurecht, denn er hat damit
wahrscheinlich dem Unternehmen über die Jahrzehnte hinweg
Milliardengewinne beschert. Aber ich bin abends mit gekrauster Stirn
nach Hause gefahren und habe gedacht, wenn ich einmal Kinder oder
Enkelkinder habe und sie fragen mich, was ich denn so in meinem Leben
gemacht habe und ich erzählen würde, das sei mein Lebenswerk, dann
verstehen die mich nicht. Das war für mich der Grund, nicht in so einem
Unternehmen zu arbeiten. Morgens in so ein Ding reingehen und abends
wieder raus und nach Jahrzehnten mühevoller Arbeit ein solches Ergebnis
rausziehen, das war mir zu wenig! Und fachlich auch zu sehr fokussiert.
Das ist an der UB Freiburg ganz anders. Vormittags habe ich einen Termin
mit einem Chemiker und danach mit einem Philosophen. Nachmittags dann
mit einem Juristen und zwischendurch noch Verwaltungsarbeiten. Es ist
mehr Vielfalt und Abwechslung drin. Es gibt sehr unterschiedliche
Situationen und Anforderungen und das macht es für mich so reizvoll, in
einer wissenschaftlichen Bibliothek zu arbeiten. Aber mit dem Geschlecht
hat das nichts zu tun.

\textbf{Seit 2008 leiten Sie nun die Universitätsbibliothek Freiburg.
Können Sie zwischen sich und männlichen Kollegen in einer ähnlichen
Position einen unterschiedlichen Führungsstil feststellen?}

Man muss ja erstmal die Frage stellen: Was ist eher weiblich und eher
männlich? Ich persönlich tue mich damit immer ein bisschen schwer. Ich
sage es mal so: Es gibt Unterschiede zwischen Männern und Frauen, auch
in Führungspositionen, die ich auch selber als typisch männlich oder
typisch weiblich wahrnehme. Ich nehme aber genauso Unterschiede zwischen
den Generationen wahr. Es gab ja hier in Baden-Württemberg durchaus ein
Generationenwechsel in den UB-Leitungen. Den habe ich miterlebt und da
hat sich vieles getan. Führungskräfte verhalten sich heute einfach
anders, legen einen anderen Stil an den Tag als sie das noch vor zwanzig
Jahren getan haben.

\textbf{Vermutlich hierarchischer?}

Hierarchischer. Allein auch durch die traditionelle, konventionelle
Arbeitsweise. Es läuft heute in aller Regel ganz anders. Das wird uns
mit team- und führungsorientierten Arbeitsstilen schon in der Schule mit
auf den Weg gegeben.

Ob es zu geschlechtspezifischen Unterschieden eine Erhebung gibt, kann
ich nicht sagen. Kollegen, die ich in unterschiedlichen Kontexten erlebe
-- in größeren Besprechungen, in Sitzungen beispielsweise in der Sektion
4 im Deutschen Bibliotheksverband -- da sehe ich starke persönliche
Ausprägungen, aber ich kann sie nicht auf das Geschlecht zurückführen.
Dennoch, ich habe natürlich auch schon ganz geschlechtsspezifische
Verhaltensweisen im positiven wie auch im negativen Kontext erlebt.

\textbf{Geschlechtsspezifische Verhaltensweisen kommen also vor, sind
aber nicht die einzigen Unterschiede. Und hinsichtlich der inhaltlichen
Gestaltung, gibt es da Unterschiede zwischen Frauen und Männern?}

Das kann ich auch nicht abschließend beurteilen und weiß auch nicht, ob
es dazu Erhebungen gibt. Ich würde sagen, technisch-orientierte
Fragestellungen könnten vielleicht eher den Männern zuzuschreiben sein
als den Frauen. Jetzt sage ich aber mal umgekehrt: Ich bin das Beispiel
in die andere Richtung. Eine stark IT-technische Ausrichtung, während
zum Beispiel ein mir bekannter männlicher Kollege als
Geisteswissenschaftler ganz andere Schwerpunkte setzt. Man müsste
tatsächlich einmal statistisch valide Erhebungen machen, um das
auswerten zu können. Mich würde interessieren, ob da etwas bei
herauskommt.

\textbf{Vielleicht wäre das etwas für eine Masterarbeit.}

Ja, genau. Ich muss auch für mich persönlich sagen: Ich hatte ja bereits
in sehr unterschiedlichen, geschlechtsspezifischen Kontexten gearbeitet.
In der Chemie gab es sehr viele Männer. In meiner Arbeitsgruppe, in der
ich promoviert habe, war ich jahrelang die einzige Frau. Mein
Doktorvater begrüßte mich damals mit den Worten: \enquote{Hallo, wir
sind hier nicht frauenfeindlich. Wir hatten schon einmal eine Frau, die
ist aber leider direkt nach dem Diplom gegangen.}

Wir haben das Thema auch letzte Woche gehabt. Da war Carl Djerassi in
Freiburg, der chemische Vater der Pille. Es ging auch um dieses Thema:
Frauen in den Naturwissenschaften. Aus heutiger Rückblende muss ich
sagen: Manches war schwer für Frauen. Es gab ja auch eine klare Aussage:
Wer seine Promotion noch nicht unter Dach und Fach hat, der braucht sich
keine Gedanken über Familie oder Nachwuchs zu machen. Das wäre damals
nahezu unmöglich gewesen und ist natürlich eine Form von
Diskriminierung. Ich habe es aber damals nicht als solche wahrgenommen.
Aus heutiger Sicht nehme ich das ganz stark wahr, aber damals war mir
das nicht bewusst. Genauso kann ich heute fragen: Wie urteile ich über
meinen Kollegenkreis in der bibliothekarischen Leitungsebene in zwanzig
Jahren in der Rückblende?

\textbf{Sie kamen also bereits aus einem männerreichen Umfeld. Also
haben Sie nicht erst als Füh\-rungs\-kraft angefangen als typisch männlich
geltende Verhaltensmuster, wie zum Beispiel Durchsetzungskraft,
Risikobereitschaft, Selbstbeherrschung\footnote{\href{http://de.wikipedia.org/wiki/M\%C3\%A4nnlichkeit}{http://de.wikipedia.org/wiki/Männlichkeit}},
zu entwickeln?}

Sie haben ja drei Beispiele genannt. Wer diese Attribute nicht hat, der
kann ein Studium in einer harten Naturwissenschaft nicht überstehen. Ich
habe diverse Kommilitonen erlebt, an denen mir sehr viel lag, die
aufgrund mangelnder Selbstbeherrschung und Durchsetzungskraft das
Studium nicht geschafft haben. Auch Beharrungsvermögen ist für mich da
ganz wichtig. Risikobereitschaft muss ich auch haben, sonst kann ich so
ein Studium nicht machen. Ich muss mir etwas zutrauen, sonst kann ich
nicht mit gefährlichen Substanzen und Apparaturen arbeiten. Wenn ich
nicht risikobereit gewesen wäre, wäre ich nicht in einen
Forschungsreaktor hineingegangen und hätte vielleicht auch die
Selbstbeherrschung verloren, als die Türen bei einem Alarm einmal
zugingen. Aber das sind letztendlich Fähigkeiten, die auch ein
Leistungssportler haben muss oder Personen, die in anderen beruflichen
Kontexten anspruchsvoll arbeiten wollen. Diese Eigenschaften werden in
der Regel Männern zugesprochen. Ich finde, sie sollten all denjenigen
Personen zugesprochen und auch von ihnen erwartet werden können, die
beruflich erfolgreich in gehobener Position arbeiten. Wenn ich diese
Eigenschaften nicht hätte, könnte ich meinen Job nicht machen und dann
wäre ich wahrscheinlich auch schnell gescheitert.

\textbf{Man liest und erlebt leider nach wie vor, dass diese
Eigenschaften geschlechtsspezifisch zugeordnet werden und nicht in
erster Linie mit verantwortungsvollen Positionen in Verbindung gebracht
werden.}

Jetzt komme ich zu einer anderen Facette. Das ist mir auch erst in den
letzten Jahren aufgefallen und bewusst geworden, so dass ich mich aktiv
mit der Thematik beschäftigt habe: Benachteiligung, Gleichberechtigung,
Frauenförderung, Frauendiskriminierung waren für mich eigentlich nie ein
aktuelles Thema. Weil ich als Führungskraft schon immer gesagt habe: Ich
benachteilige und bevorzuge niemanden. Ich fördere die, die gute Arbeit
machen. Das ist für mich das Entscheidende. Dann habe ich aber durchaus
Situationen erlebt, in denen ich mich schlecht behandelt gefühlt habe.
Wo ich nicht so richtig zuordnen konnte: Was passiert hier eigentlich?
Das war interessanterweise vor ein paar Jahren ein Kollege -- das war
übrigens nicht im bibliothekarischen Bereich~ --, den ich um einen Rat
gebeten habe, weil ich nicht verstehen konnte, was da gerade passiert.
Er sagte: \enquote{Das was hier passiert, ist ein Geschlechterproblem.
Hier sind Männer, die haben ein Problem mit Frauen in Führungspositionen
und in persona mit Ihnen. Ich gebe Ihnen mal ein Buch mit, lesen sie
das.} Das war \enquote{Das Arroganzprinzip} von Peter Modler\footnote{\url{http://www.fischerverlage.de/buch/das_arroganz-prinzip/9783596184330}}.
Das ist ein Unternehmensberater, der seit vielen Jahren Beratungen und
Workshops für Frauen in Führungspositionen anbietet. Er schildert in
diesem Buch Berichte von Frauen, die Diskriminierung in ihrem
Berufsalltag erlebt haben. Da sind mir beim Lesen viele Lichter
aufgegangen. Seitdem suche ich auch aktiv das Gespräch mit Frauen in
Leitungspositionen und stelle fest, die haben Ähnliches erfahren und
haben es ebenfalls nicht bewusst zugeordnet. Es gibt also sehr wohl
solche Anzeichen, auch in unserer Branche, mit denen man sehr achtsam
umgehen muss. Wo man Frauen auch die notwendige Sensibilisierung mit auf
den Weg geben und sie darin stärken muss, sich das nicht gefallen zu
lassen. Ich bin seitdem öfter darauf eingegangen und habe reagiert. Wenn
mich jemand angeblökt hat, habe ich eben zurückgeblökt. Davor war ich
immer auf die Sache orientiert und wollte Konsens herbeibringen, bin
ruhig geblieben. Aber nachdem ich das gelesen hatte, habe ich mir
gedacht: Was habe ich zu verlieren? Soll ich mich wieder so behandeln
lassen? Jetzt wehre ich mich mal! Und von einem Tag zum anderen war Ruhe
und ich habe keine Probleme mehr gehabt. Da habe ich gemerkt, dieses
Thema existiert durchaus in unserem Berufsleben.

\textbf{Man muss es aber erst einmal wahrnehmen.}

Genau. Oft sitzt man ja mittendrin und kann es gar nicht erkennen. Das
kann ich erst, wenn ich beobachte und weiß, was ist mein Suchraster,
worauf muss ich achten. Erst dann kann ich es feststellen.

\textbf{Meiner eigenen Beobachtung zufolge gibt es in den öffentlichen
Bibliotheken mehr weibliche Führungs\-kräfte als in den wissenschaftlichen
Bibliotheken. Sind Ihnen statistische Zahlen zum Frauenanteil in
bibliothekarischen Leitungspositionen bekannt?}

Ich habe in Vorbereitung auf das Gespräch ein wenig gegoogelt und eine
Grafik über Führungs\-positionen gefunden, die ganz frappierend ist.
Insgesamt ist der Frauenanteil in Bibliotheken durch alle Ebenen hinweg
nach einer Erhebung des IAB im Bibliotheksbereich bei fast 75\%. Diese
Grafik\footnote{\url{http://infobib.de/blog/2013/03/18/frauen-in-fuhrungspositionen-bei-dbv-mitgliedern/}}
ist nun anhand des Adressverzeichnisses des Deutschen
Bibliotheksverbands erstellt worden. Es wurde, auf die Sektionen des DBV
aufgedröselt, der Anteil an Frauen in Führungs\-positionen ermittelt. Beim
Führungspersonal in den öffentlichen Bibliotheken korreliert der Anteil
an Männern mit der Größe der Bibliothek. In den wissenschaftlichen
Bibliotheken liegt der Anteil der Frauen, wie bei den
Großstadtbibliotheken, bei etwa 50\% und damit deutlich niedriger als in
den anderen Sektionen.

\textbf{Frauen sind also, gemessen an ihrem Gesamtanteil im
Bibliothekswesen, deutlich geringer in Führungspositionen vertreten.
Woran könnte das liegen?}

Tja, ganz spontan hätte ich auch vermutet, dass es mit der Größe der
Einrichtung korreliert. Ähnlich ist es ja auch in der Wirtschaft. Bei
der fachlichen Ausrichtung, also ÖB oder WB, stelle ich mir die Frage:
Wer sitzt da in den Auswahlgremien? Wenn ich Universitäten betrachte,
aber auch viele Hochschulen, sind Leitungsebenen und Rektorate oft noch
in hohem Maße von Männern dominiert. In Freiburg haben wir gerade die
zweite Prorektorin bekommen, bei einer Universitätsgeschichte von 557
Jahren! Man weiß ja auch aus empirischen Erhebungen in den
unterschiedlichsten Branchen, dass Männer geneigt sind, bevorzugt Männer
einzustellen. Das könnte hier auch ein Grund sein. Je größer die
Kommune, desto stärker ist die kommunale Leitungsebene männerdominiert
und im Hochschulbereich ist es eben ganz ähnlich. Wir haben in vielen
Studiengängen ja inzwischen einen Frauenüberschuss bei den Studierenden,
aber im Lehrpersonal ist der Männerüberschuss da -- bis hin zu
Fakultäten, wo nur eine oder zwei Frauen im Lehrkörper sind.

\textbf{Das dürfte sich über die Generationen hinweg etwas auflösen.}

Ja, ich habe eine zweite Erhebung gefunden. Es gibt an der Hochschule
für Technik und Wirtschaft in Chur eine Bachelorarbeit \enquote{Die
Gläserne Decke in Schweizer Bibliotheken}\footnote{\url{http://www.htwchur.ch/uploads/media/CSI_53_Stadler.pdf}},
fand ich auch ganz spannend. Da hat man auch Erhebungen aus dem
internationalen Bereich zitiert -- in dem es übrigens sehr ähnlich
aussieht --, in denen man sieht, dass es sich über die Jahre hinweg
entwickelt. Der Anteil der Frauen geht nach oben, auch in den
Führungsebenen. Insgesamt würde ich sagen, sieht es für die Frauen im
bibliothekarischen Bereich immer noch gut aus. Wir haben insgesamt einen
hohen Anteil an Frauen in der Bibliotheksbranche und der Anteil an
Führungspositionen ist im Vergleich zu anderen Branchen auch hoch. Ich
würde vermuten, dass im Schulbereich die Diskrepanz zwischen dem Anteil
an Lehrerinnen und Schuldirektorinnen größer ist.

\textbf{Also sieht es doch gar nicht so schlecht aus.}

Finde ich auch. Und wenn man einmal zurückschaut, wie lange Frauen
überhaupt im Bibliothekswesen arbeiten, dann hat sich das durchaus gut
entwickelt. Bei uns ganz konkret -- ich bin gerade mit unserem
Referendar die Stellenpläne durchgegangen -- haben wir auch einen
deutlichen Frauenüberschuss in der Belegschaft der UB Freiburg. Das
führt dazu, dass unsere Gleichstellungsbeauftragte der Universität bei
Stellenbesetzungsverfahren häufig abwinkt. Ausnahme: Im höheren Dienst,
ab Gruppe 13, will sie dabei sein, denn da haben wir noch keinen
gleichwertigen Frauenanteil.

In der Churer Arbeit gibt es auch viele Erhebungen in Form von
Fragebögen und Interviews, die sich mit Frauen in Leitungspositionen
beschäftigen. Eine Herausforderung, die wir auch aus anderen Branchen
kennen, ist die Vereinbarkeit von Beruf und Familie. Die Verantwortung
für die Familie liegt nach wie vor im hohen Maße bei der Frau. Das führt
dazu, dass die Familienplanung im Studium nach hinten geschoben wird.
Dann kommt der Berufseinstieg und dann geht man schon langsam auf die 40
zu und es wird Zeit für den Nachwuchs. Genau zu diesem Zeitpunkt ist man
im Beruf aber in einem Alter, in dem es richtig spannend wird für höhere
Tätigkeiten, qualifizierte Leitungspositionen. Das kollidiert mit der
Familienplanung. Das dürfte ein ganz wesentlicher Faktor sein. Da gehen
leider noch viele Arbeitgeber nicht drauf ein. Das sind manchmal ganz
subtile Methoden: Dienstbesprechungen, die abends um 19 Uhr angesetzt
werden, wo doch jeder wissen sollte, dass für diese Zeit eine
Kinderbetreuung nicht ohne Weiteres auf die Beine gestellt werden kann.
Die Erwartung, 70 Stunden pro Woche im Büro ansprechbar zu sein. Das war
auch für mich ein Lernprozess. Dass ich heute mal meinem Rektor sage:
\enquote{Jetzt muss ich mich um meine Familie kümmern!}, das hat bei mir
auch gedauert.

\textbf{Dieses Bedürfnis zu kommunizieren und sich dieses Recht selbst
zuzugestehen.}

Genau. Das weiß man ja auch und da mache auch ich mich nicht von frei:
Frauen haben für sich immer den Anspruch, doppelt so viel leisten zu
müssen, um gleichberechtigt zu einem Mann bewertet zu werden. Ich
behaupte für mich als Führungskraft, das ist zumindest mein herer
Anspruch, dass ich keinen Unterschied zwischen Mitarbeiterinnen und
Mitarbeitern mache.

\textbf{Die UB Freiburg wird zur Zeit saniert und voraussichtlich im
Wintersemester 2014/15 neu eröffnet. Sie waren in den letzten Jahren
also auch stark im Bauumfeld involviert, und damit erneut in einem sehr
männerdominierten Bereich. Haben Sie Ihre Weiblichkeit als in
irgendeiner Weise nachteilhaft empfunden?}

Das kenne ich aus dem beruflichen und privaten Bereich. Ich habe vor
wenigen Jahren mit meinem Mann zusammen ein Haus gebaut. Im privaten
Umfeld war es für mich oft schlimmer. Wenn die typischen Handwerker um
die Ecke kamen, die sich von einer Frau ungerne etwas sagen lassen
wollten. Im dienstlichen Umfeld ist es mir bisher selten passiert und
dann habe ich es pariert, indem ich sachlich und fachlich informiert
war. So kann ich mich an eine Diskussion erinnern, bei der es
stundenlang hoch her ging: Welche klimatechnischen Voraussetzungen
braucht der Sonderlesesaal? Und ich bemerkte -- das ist ja oft so bei
solchen Baubesprechungen -- man ist wirklich umgeben von Männern. Die
hatten keine Lust zusätzliches Geld in die Hand zu nehmen und es lag auf
der Hand, da war ein Planungsfehler gemacht worden. Die hatten die
Vorstellung: Dann wird es halt mal 30 Grad im Sonderlesesaal und man
kann den Nutzern halt vier Wochen lang keine mittelalterliche
Handschrift zur Verfügung stellen. Ich habe dann gesagt: \enquote{Wir
müssen nach DIN-Fachbericht und den Normen gehen und das muss gemacht
werden.} Ich konnte das auch taktisch sehr leicht aushebeln, denn unser
Rektor ist Mediävist. Da sagte ich: \enquote{Dann erklären Sie unserem
Rektor, dass er im Sommer vier Wochen lang nicht arbeiten kann.} Aber
Ruhe geschaffen und eine Lösung erzielt habe ich eigentlich dadurch,
dass ich mich selber hingesetzt und die Normen gelesen habe. So dass ich
rezitieren konnte, was meine Anforderungen sind. Als ich das vorgelesen
habe, waren die Bauingenieure und Architekten -- die diese Normen nicht
angeguckt hatten, weil sie nicht damit gerechnet hatten, dass ich als
Bibliothekarin sie lesen würde -- sprachlos und hatte dem nichts
entgegen zu setzen. Und dann wurde die zusätzliche Lüftungsanlage
gebaut.

\textbf{Sie haben sie also durch fachliche Argumente überzeugen können.}

Ja. Jetzt kann ich aber nicht genau sagen: Wollten die das auf die
einfache Tour mit mir durchziehen, weil ich eine Frau war? Oder weil ich
ein Bibliothekar war, der von Technik keine Ahnung hat?

\textbf{Ja, es bleibt unklar, welches Klischee da wohl eine Rolle
gespielt hat. Ertappen Sie sich denn manchmal dabei, dass Sie mit Ihren
männlichen Mitarbeitern anders umgehen als mit Ihren Mitarbeiterinnen?}

Auf unbewusster Ebene kann sich davon, glaube ich, niemand frei machen.
Wer das behauptet, ist nicht hinreichend selbstkritisch. Frauen gehen
mit Frauen anders um, als mit Männern und umgekehrt auch. Ich habe für
mich den Anspruch, damit so professionell umzugehen, dass ich bewusst
keinen Unterschied mache. Ich muss ja auch sowohl mit Menschen
zusammenarbeiten, die ich gut leiden kann und die mir menschlich
sympathisch sind, als auch mit Menschen, die ich nicht so gut leiden
mag. Trotzdem muss ich zu jeder Zeit, auf jeder Ebene fachlich
konstruktiv und professionell mit ihnen zusammenarbeiten. Das gilt für
meine Mitarbeiter und meine Kollegen genauso wie für meine Vorgesetzen.

Wo ich vielleicht aktiv und bewusst einen Unterschied mache, ist, wenn
ein männlicher Mitarbeiter zu mir kommt und sagt: Ich möchte mich jetzt
um meine Familie kümmern. Solche Anträge haben wir in der letzten Zeit
des Öfteren auf dem Tisch gehabt und da sage ich freudestrahlend: Prima,
das unterstützen wir! Denn nur so werden wir auch wirklich etwas
erreichen können. Wenn Männer sich stärker am Familienleben beteiligen
und dafür auch mal beruflich zurückstecken und Kompromisse schließen.
Damit gehe ich also umgekehrt bevorzugend um und da stehe ich auch dazu.

\newpage 

\textbf{Wen empfinden Sie als ein positives Beispiel im Bibliothekswesen
und warum?}

Das ist Elisabeth Niggemann, Generaldirektorin der Deutschen
Nationalbibliothek. Sie hat auch Familie und hat es mit ihrem Beruf
unter einen Hut gebracht. Sie hat Ihre Karriere zielstrebig und in
jungen Jahren sehr weit vorangetrieben, weiter kann man es in
Deutschland ja nicht schaffen. Aber ich kann jetzt auch gar nicht alle
nennen. Ich treffe viele Menschen, deren Arbeit ich sehr schätze, die
ich auch als Person schätze und gerne und intensiv mit ihnen
zusammenarbeite. Deshalb würde ich den Lob auch viel lieber noch weiter
spannen und sagen: Für mich sind die wirklich hervorzuhebenden Menschen
die Frauen, die Bibliotheken überhaupt professionell mit Frauen bestückt
haben, wie Bona Peiser oder ähnliche Personen auf dem internationalen
Sektor. Es ist ja noch nicht lange so, dass Frauen hier überhaupt
arbeiten können und sich ausbilden dürfen. Und ansonsten sind das für
mich Menschen, die auch heute in schwierigen politischen Situationen
ihren Beruf ausüben müssen. Denn wer in einer Bibliothek arbeitet,
Wissen und Informationen verwaltet --~ das hat schon das Dritte Reich
gezeigt -- , ist für totalitäre Regime per se angreifbar und steht immer
an vorderster Front als Opfer. Dass es da Leute gibt, die das trotz der
widrigen Umstände machen, die \emph{(lacht)} ihren Mann stehen -- da
sieht man wieder an unserer Alltagssprache, wie wir doch eben noch in
einer männlichen Berufswelt leben --, denen würde ich das Wort Helden
zuordnen. Obwohl ich noch nicht einmal Namen nennen kann. Denn das ist
ja auch typisch für das Berufsfeld, Helden wie im Sport und in der
Politik haben wir eben doch nicht. Mit Bibliotheken verbindet man im
Allgemeinen nicht Personen des öffentlichen Lebens. In einer städtischen
oder regionalen Öffentlichkeit ist das vielleicht etwas anderes.

\textbf{Zum Abschluss möchte ich Sie noch fragen: Was sind Ihre
persönliche Empfehlungen für eine junge Schulabgängerin, die sich
beruflich ins Bibliothekswesen hin orientieren möchte?}

Für mich war immer Grundprinzip -- und das sage ich jedem jungen
Menschen --: Wenn man Leidenschaft für etwas empfindet, etwas hat, das
einen interessiert und das man spannend findet, was man gerne machen,
lernen und wissen möchte, dann soll man genau das machen. Egal, wie die
Umstände sind. Denn wenn man es mit Leidenschaft tut und eine gewisse
Begabung dafür hat, findet man seinen Weg. Ob es eine große Karriere
ist, die einen glücklich machen muss, ist ja die zweite Frage.

Was ich häufig in Bewerbungsschreiben lese: \enquote{Ich möchte in der
Bibliothek arbeiten, weil ich so gerne lese.} Das greift mir zu kurz!
Man muss sich mit dem modernen Berufsbild, mit den Perspektiven, mit
denen wir uns auseinanderzusetzen haben, beschäftigen und für sich
klären, ob das eine Tätigkeit ist, die einem Spaß macht und die man sich
für die nächsten Jahre vorstellen kann. Man muss vor allen Dingen zur
Kenntnis nehmen, anders als vor fünfzig oder zwanzig Jahren: Dieses
Berufsbild ist einer hohen Veränderung unterworfen. Und wenn ich
Karriere machen will, muss ich zielstrebig, engagiert und realistisch
daran arbeiten. Muss dafür aktiv etwas tun und Einsatz zeigen. Mir etwas
zutrauen, Netzwerke aufbauen und sich stetig fort- und weiterbilden.
Schließlich: Keine zu langen Berufspausen einlegen, gerade bei höheren
Positionen. Das bedeutet auch, man muss sich im privaten Bereich so
aufstellen, dass es möglich ist.

\textbf{Das bringt uns wieder zum Anfang zurück: Den Menschen als
Arbeitskraft auch in seinem Umfeld betrachten.}

Genau.

\textbf{Liebe Frau Kellersohn, vielen Dank für das ausführliche Gespräch
und Ihre Offenheit.}

%autor

\end{document}